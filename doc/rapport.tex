\documentclass{article}
\usepackage{graphicx}
\usepackage[margin = 2cm]{geometry}
\usepackage{caption}
\usepackage{subcaption}

\title{PCO Laboratoire 5 \\
\large Modélisation par moniteur de Mesa}
\author{Benoît Delay, Eva Ray}

\begin{document}
\maketitle

\section*{Description des fonctionnalités du logiciel}

Le programme modélise une boutique de barbier. La boutique est constituée d'un salon dans lequel on trouve une salle d'attente avec
un nombre de siège fixes et une chaise de travail sur laquelle un client peut se faire couper les cheveux. Un certain nombre de clients
vont tenter de se faire couper les cheveux par le barbier. Le programme est doté d'une interface graphique, qui est animée grâce aux
annimations présentes dans la clase \texttt{PcoSalon} qui représentent les actions du barbier et des clients. \\

Les fonctionnalités du salon sont assez représentatives d'un salon de coiffure classique. Le salon est géré par un barbier qui s'occupe
de couper les cheveux des clients sur la chaise de travail. Lorsqu'il a fini de s'occuper d'un client, le barbier appelle le prochain client
parmi ceux qui sont en train d'attendre dans la salle d'attente. Le barbier fait bien attention à appeler les clients dans leur ordre
d'arrivée. S'il n'y a plus aucun client dans la salle d'attente, le barbier va faire une petite sieste et ne revient que lorsqu'un nouveau
client arrive à la porte de la boutique. Le comportement du barbier est simulé par la méthode \texttt{Barber::run()}. \\

Les clients, quant à eux, ont aussi un comportement prédéfini. Les clients qui souhaitent se faire couper les cheveux ne peuvent entrer
dans le salon que si celui-ci a assez de place pour les acueillir, c'est-à-dire s'il y a de la place dans la salle d'attente ou si aucun
client n'est présent dans le salon. Dans le second cas, la barbier est en train de faire une petite sieste, comme mentionné ci-dessus.
L'arrivée du nouveau client à la porte va réveiller le barbier qui se déplace jusqu'à la chaise de travail, où le nouveau client
va le rejoindre. Si le salon est plein, le
client va faire un petit tour avant de revenir voir si une place s'est libérée. Si le client accède à la salle d'attente, il va gentiment
s'asseoir sur une chaise libre et attend que le barbier l'appelle. Lorsque c'est le tour du client, il se dirige vers la chaise de travail,
où le barbier lui coupe les cheveux. Une fois la coupe finie, le client quitte le salon et attend que ses cheveux repoussent avant de
revenir se faire faire une nouvelle coupe. Le comportement d'un client est simulé par la méthode \texttt{Client::run()} \\

Les comportements décrits ci-dessus sont représentés par les méthodes de la clase \texttt{PcoSalon} et correspondent aux machines d'états
illustrées ci-dessous. \\

La terminaison du programme est gérée grâce aux méthodes \texttt{PcoSalon::endService()} et \\ \texttt{PcoSalon::isInService()}.
La fin du programme
est innitiée dès qu'une entrée utilisateur est déclenchée dans le terminal. Le salon est alors fermé. Les clients qui ne se trouvent pas
actuellement dans le salon rentrent chez eux. Le barbier finit de s'occuper de tous les clients qui sont encore dans le salon mais n'en
accepte pas de nouveau. Une fois qu'il a terminé son travail, ferme le salon et termine sa journée. \\

Le logiciel simule le comportement du barbier et des clients par des threads différents, il est donc multi-threadé. Par conséquent, il doit 
assurer une bonne gestion de la concurrence pour les ressources partagées entre plusieurs threads. Les méthodes du salon sont ainsi implémentées 
à l'aide d'un moniteur de Mesa, qui permet de gérer la concurrence et les animations. 

\begin{figure}
    \centering
        \begin{subfigure}[b]{0.4\textwidth}
        \centering
        \includegraphics[width=\textwidth]{figures/machine_etat_client.png}
        \caption{Machine d'état du client}
        \label{fig:machine état client}
    \end{subfigure}
    \hfill
    \begin{subfigure}[b]{0.4\textwidth}
        \centering
        \includegraphics[width=\textwidth]{figures/machine_etat_barbier.png}
        \caption{Machine d'état du barbier}
        \label{fig:machine état barbier}
    \end{subfigure}
\end{figure}

\section*{Choix d'implémentation}
% Comment avez-vous abordé le problème, quels choix avez-vous fait, quelle 
% décomposition avez-vous choisie, quelles variables ont dû être protégées, ...

\subsection*{Variables de condition}

La synchronisation entre les différents threads du programme se fait via des variables de condition qui sont des attributs
protégés de la classe \texttt{PcoSalon}.

\begin{itemize}
    \item \texttt{canGoForHaircut}: La variable de condition \texttt{canGoForHaircut} représente l'attente d'un client sur une des chaises 
    de la salle d'attente avant de pouvoir aller se faire couper les cheveux. Les threads des clients attendent sur cette variable de 
    condition dans la méthode \texttt{PcoSalon::accessSalon} et le barbier réveille les threads des clients en attente depuis la méthode 
    \texttt{PcoSalon::pickNextClient}.
    \item \texttt{waitClient}: La variable de condition \texttt{wairClient} représente le fait que le barbier attente qu'un client se déplace
    de la salle d'attente à la chaise de travail. Le thread du barbier attent sur cette variable de condition dans la méthode \texttt{PcoSalon::waitClientAtChair} et le 
    client signifie qu'il a atteint la chaise de travail dans la méthode \texttt{PcoSalon::goForHaircut}.
    \item \texttt{doneBeautified}: La variable de condition \texttt{doneBeautified} représente le fait qu'un client attendre que sa coupe de
    cheveux se termine. Le thread du client attent sur cette variable de condition dans la méthode \texttt{PcoSalon::goForHaircut} et le 
    barbier signifie au client qu'il a finit la coupe dans la méthode \\ \texttt{PcoSalon::beautifyClient}. 
    \item \texttt{barberSleeping}: La variable de condition \texttt{barberSleeping} représente le sommeil du barbier. Le thread du barbier attent sur cette variable 
    de condition dans la méthode \texttt{PcoSalon::goToSleep} pour représenter la fait que le barbier s'endort. Le 
    client signifie que le barbier doit se réveiller dans la méthode \\ \texttt{PcoSalon::accessSalon}.
\end{itemize}

\subsection*{Gestion de la file FIFO}

Une des fonctionnalités du programme est d'assurer le fait que le barbier s'occupe des clients dans leur ordre d'arrivée dans le salon. Pour ce faire,
nous avons implémenter une file FIFO avec une méthode de ticketing. L'idée est simple; chaque client qui arrive dans la salle d'attente prend un ticket à
la machine à tickets, représentée par l'attribut \texttt{ticketMachine} de la classe \texttt{PcoSalon}. Il attend ensuite sur la variable de condition 
\texttt{canGoForHaircut}. Lorsqu'il est prêt, le barbier va \texttt{notifyALL} sur cette variable, ce qui va avoir pour conséquence de réveiller tous les 
threads en attente. A ce moment, le client réveillé va comparer son numéro de ticket avec celui stocké dans l'attribut \texttt{nextClient} de la 
classe \texttt{PcoSalon} qui représente le numéro du prochain client à pouvoir aller se faire couper les cheveux. Si les numéros coïncident, le client
peut aller se faire couper les cheveux et l'attribut \texttt{nextClient} est incérmenté, sinon, il se remet en attente jusqu'à ce que ce soit son tour.  

\subsection*{Gestion de l'interruption du programme}

La terminaison du programme est amorcée lorsqu'une entrée utilisateur est déclenchée dans le terminal. Elle est gérée grâce aux méthodes 
\texttt{PcoSalon::endService()} et \texttt{PcoSalon::isInService()}. Nous avons ajouté un attribut booléen \texttt{isOpen} dans la classe \texttt{PcoSalon} 
qui vaut \texttt{true} lorsque le salon est ouvert et \texttt{false} sinon. La méthode \texttt{PcoSalon::isInService()} renvoie simplement cet attirbut. Elle nous
permet de connaître l'état d'ouverture du salon depuis l'extérieur de la classe. Dans la méthode \texttt{PcoSalon::endService()} on change la valeur de 
\texttt{isOpen} à \texttt{false} pour informer de la fermeture du salon. On s'assure aussi de gérer le cas dans lequel la fermeture du salon est 
déclenchée lorsque le barbier est en train de dormir. Dans ce cas, on réveille le barbier pour que son thread puisse se finir et ne pas rester en 
attente indéfiniment sur la variable de condition \texttt{barberSleeping}. 
On s'assure ensuite dans la méthode \texttt{Barber::run} du barbier qu'il s'occupe des clients qui sont étaient déjà présents dans la salon avant la
fermeture avant que son thread ne s'arrête pour de bon. Cela est mis en place grâce aux méthodes \texttt{PcoSalon::isInService()} et
\texttt{PcoSalon::getNbClient()}. Les clients attendent encore que leur cheveux repoussent avant de se rendre compte que le salon est fermé et rentrer 
chez eux pour de bon.

\section*{Tests effectués}
% Description de chaque test, et information sur le fait qu'il ait passé ou non

Le résultat des tests ci-dessus a notamment pu être vérifié car nous logons dans l'interface graphique des informations sur la plupart des 
actions entreprises par le barbier et les clients. 

\subsection*{Test: Fonctionnalités du salon}

\begin{itemize}
    \item 
\end{itemize}

\subsection*{Test: Terminaison du programme}

\section*{Conclusion}

Nous avons implémenté un programme multi-threadé qui simule un salon de coiffure conformément au cahier des charges. Nous avons implémenté une
gestion de la synchronisation grâce à des variables de condition d'un moniteur de Mesa qui semblent faire leur travail convenablement. 
L'accès des clients au salon, l'attente dans la salle d'attente, la prise en charge des clients, la coupe, la repouss des cheveux et le 
sommeil du barbier sont gérés correctement.
L'innitation de la fin de programme par le déclenchement d'une entrée utilisateur dans le terminal fonctionne comme espéré et permet de 
mettre fin aux threads de la simulation correctement.

\end{document}